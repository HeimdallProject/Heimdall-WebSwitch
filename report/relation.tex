% --- Documentclass specifications ---
\documentclass[italian]{tktltiki2}
\linespread{1.3}

% --- General packages ---
\usepackage[utf8]{inputenc}
\usepackage[T1]{fontenc}
\usepackage{lmodern}
\usepackage{microtype}
\usepackage{amsfonts,amsmath,amssymb,amsthm,booktabs,color,enumitem,graphicx}
\usepackage[pdftex,hidelinks]{hyperref}
% Automatically set the PDF metadata fields
\makeatletter
\AtBeginDocument{\hypersetup{pdftitle = {\@title}, pdfauthor = {\@author}}}
\makeatother


% --- Language-related settings ---
\usepackage[fixlanguage]{babelbib}
% add bibliography to the table of contents
\usepackage[nottoc]{tocbibind}


% --- tktltiki2 options ---
%
% The following commands define the information used to generate title and
% abstract pages. The following entries should be always specified:
\title{%
  \huge Project Heimdall \\
  \large Proposta di implementazione per un webswitch \\ 
    concorrente two-way di livello 7 (OSI) con \\
    politiche di bilanciamento del carico stateless e stateful
  }
\author{\emph{Alessio Moretti} - 0187698 \\\emph{Andrea Cerra} - ??????\\\emph{Claudio Pastorini} - 0186256}
\date{\today}
\level{Corso di Ingegneria di Internet e del Web - A.A. 2015/2016}
\university{\textbf{Università di Tor Vergata}}
\department{\textbf{Ingegneria Informatica}}
\city{Roma}

% If the automatic page number counting is not working as desired in your case,
% uncomment the following to manually set the number of pages displayed in the abstract page:
%
% \numberofpagesinformation{16 pages + 10 appendix pages}

%


\begin{document}

% --- Front matter ---
\maketitle        % title page

\tableofcontents  % table of contents
\pagenumbering{gobble}

% --- Main matter ---
\mainmatter       % clear page, start arabic page numbering

\section{Example section}
% Write some science here
Sample text and a reference\cite{lamport94}. Lorem ipsum dolor sit amet, consectetur adipiscing elit. Donec at lorem varius, sodales diam semper, congue dui. Integer porttitor felis eu tempor tempor. Proin molestie maximus augue in facilisis. Phasellus eros dui, blandit eu nibh ut, pharetra porta enim. Cum sociis natoque penatibus et magnis dis parturient montes, nascetur ridiculus mus. Aliquam ullamcorper risus pretium est elementum, eget egestas lorem fermentum. Etiam auctor nisi purus, vitae scelerisque augue vehicula sed. Ut eu laoreet ex. Mauris eu mi a tortor gravida cursus eget sit amet ligula.

\begin{figure}
\centering
\includegraphics[width=\textwidth]{images/thor}
\caption{Thor di Asgard, \emph{figlio di Odino}}
\end{figure}



\newpage
\section{Introduzione}

\subsection{Perchè Heimdall?}
\subsection{Web-Switch di livello 7}



\newpage
\section{Architettura}

\subsection{Server in ascolto}
\subsubsection{File di configurazione}
\subsubsection{Logging}
\subsubsection{Gestione degli errori}

\subsection{Pool manager}

\subsection{Scheduler}

\subsection{Worker}
\subsubsection{Gestione delle richieste}
\subsubsection*{Coda delle richieste}
\subsubsection*{Chunk di dati}
\subsubsection{Gestione delle connessioni}
\subsubsection*{Connessione}
\subsubsection*{Richieste HTTP}
\subsubsection*{Risposte HTTP}
\subsubsection{Thread di lettura}
\subsubsection{Thread di scrittura}
\subsubsection{Thread di richiesta}
\subsubsection{Thread di watchdog}

\newpage
\section{Ulteriori proposte}

\newpage
\section{Politiche di scheduling}
\subsection{State-less: implementazione con Round Robin}
\subsection{State-aware: implementazione con monitor di carico}
\subsubsection{Modulo ApacheStatus}

\newpage
\section{Performance}
\subsection{Test di carico}
\subsection{Comparazione con Apache}

\newpage
\section{Future implementazioni}
\subsection{Webserver performante}


% --- References ---
%
% bibtex is used to generate the bibliography. The babplain style
% will generate numeric references (e.g. [1]) appropriate for theoretical
% computer science. If you need alphanumeric references (e.g [Tur90]), use
%
% \bibliographystyle{babalpha-lf}
%
% instead.


\newpage
%\bibliographystyle{babplain-lf}
%\bibliography{references}
\renewcommand{\refname}{\normalfont\selectfont\normalsize\textbf{Annotazioni}} 
\begin{thebibliography}{9}
\bibitem{lamport94}
  Leslie Lamport,
  \emph{\LaTeX: a document preparation system},
  Addison Wesley, Massachusetts,
  2nd edition,
  1994.
  
\end{thebibliography}

% --- Appendices --- 
\newpage
\appendix
 
\section{Manuale per l'uso}

\section{Vagrant}

\section{Cluster virtuale}

\section{Tool per i debug}
\subsection{GDB}
\subsection{Valgrind}

\section{Tool per i test}
\subsection{PostMan}
\subsection{Telnet}
\subsection{HttPerf}
\subsection{Browser}

\end{document}